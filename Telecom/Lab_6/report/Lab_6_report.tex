\include{settings}

\begin{document}	% начало документа

% Титульная страница
\include{titlepage}

% Содержание
\include{ToC}


\section{Цель работы}
Изучить основные возможности библиотеки gnuradio и разработать тестовый пример, с использованием библиотеки gnuradio.

\section{Программа работы}
В среде gnuradio реализовать программу, включающую:
						\\-генератор зашумленного сигнала
						\\-фильтр низких частот, для фильтрации шума
						\\-блок, рассчитывающий mae(mean absolute error)
						\\-графический интерефейс, позволяющий анализировать качество фильтрации, а также, позволяющий изменять параметры фильтра.


\section{Ход выполнения работы}


\subsection{Листинг (Блок-схема)}

Variable - блок-переменная.

QT GUI Range - блок, создающий переменную и графический интерфейс, позволяющий изменять значение созданной переменной в процессе работы.

Signal Source - блок, генерирующий сигнал с заданными параметрами.

Noise Source - блок, генерирующий шум с заданными параметрами.

Subtract, Add, Abs - блоки, выполняющие арифметические операции.

Throttle - блок, с заданной частотой дискретизации преобразовывающий исходный аналоговый сигнал в дискретный.

Low Pass Filter - блок-фильтр нижних частот.

QT Gui Time Sing - блок, создающий графический интерфейс для отображения сигналов во времени.


\begin{figure}[H]
	\begin{center}
		\includegraphics[scale=0.7]{block_schema}
		\caption{Блок-схема}
		\label{pic:block_schema} % название для ссылок внутри кода
	\end{center}
\end{figure}

\subsection{Исходный зашумленный сигнал}
\begin{figure}[H]
	\begin{center}
		\includegraphics[scale=0.7]{noise_init}
		\caption{Исходный зашумленный сигнал}
		\label{pic:noise_init} % название для ссылок внутри кода
	\end{center}
\end{figure}
\newpage

% ------------------------------- init block -------------------------------------------
\subsection{Результаты фильтрации до настройки фильтра}
\begin{figure}[H]
	\begin{center}
		\includegraphics[scale=0.5]{graphic_init}
		\caption{Исходный и отфилтрованный зашумленный сигнал}
		\label{pic:graphic_init} % название для ссылок внутри кода
	\end{center}
\end{figure}

\begin{figure}[H]
	\begin{center}
		\includegraphics[scale=0.5]{mae_init}
		\caption{Средняя абсолютная ошибка}
		\label{pic:mae_init} % название для ссылок внутри кода
	\end{center}
\end{figure}
\newpage
% ------------------------------- edited block 1 -------------------------------------------
\subsection{Результаты филтрации с хорошо подоборанным параметром transition width}
\begin{figure}[H]
	\begin{center}
		\includegraphics[scale=0.5]{good_params}
		\caption{Значение параметра transation width}
		\label{pic:good_params} % название для ссылок внутри кода
	\end{center}
\end{figure}


\begin{figure}[H]
	\begin{center}
		\includegraphics[scale=0.5]{good_graphics}
		\caption{Исходный сигнал, отфилтрованный зашумленный сигнал и средняя абсолютная ошибка}
		\label{pic:good_graphics} % название для ссылок внутри кода
	\end{center}
\end{figure}
\newpage

% ------------------------------- edited block 2 -------------------------------------------
\subsection{Результаты филтрации с плохо подоборанным параметром transition width}
\begin{figure}[H]
	\begin{center}
		\includegraphics[scale=0.5]{bad_params}
		\caption{Значение параметра transation width}
		\label{pic:bad_params} % название для ссылок внутри кода
	\end{center}
\end{figure}


\begin{figure}[H]
	\begin{center}
		\includegraphics[scale=0.5]{bad_graphics}
		\caption{Исходный сигнал, отфилтрованный зашумленный сигнал и средняя абсолютная ошибка}
		\label{pic:bad_graphics} % название для ссылок внутри кода
	\end{center}
\end{figure}
\newpage

% ------------------------------- init block -------------------------------------------

\section{Выводы}

В данной работе были изучены основные возможности библиотеки gnuradio, была произведенна фильтрация зашумленного сигнала и интерактивная настройка фильтра в графической среде библиотеки gnuradio.
\end{document}
